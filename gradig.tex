\chapter{Grading-System with If}

\begin{minted}[frame=lines,framerule=3pt, framesep=10pt, fontsize=\small, linenos]{matlab}
% Internal Full Marks is 20 and final full marks is 30
I=input('Internal Marks Obtained:');
F=input('Final Marks Obtained:');

R=min(I,F/30*20*1.2);
fprintf('Revised Internal Marks is %2.2f.\n',R);

T=R+F;
fprintf('Total Marks Obtained is %2.2f.\n',T);

if I>20||F>30
  disp('Wrong Information.');
else
  if T>=45
    y='A';
  elseif T>=40
    y='A-';
  elseif T>=35
    y='B';
  elseif T>=30
    y='B-';
  else
    y='Fail';
  end
end
fprintf('The obtained grade is %s.\n',y)

\end{minted}
\clearpage

\chapter{Circle, Ellipse, Hyperbola, Parabola}

\subsection{Circle}
\begin{minted}[frame=lines,framerule=3pt, framesep=10pt, fontsize=\small, linenos]{matlab}
r = input('Enter the value of radius:');
h = input('Enter x-coordinate of center:');
k = input('Enter y-coordinate of center:');

theta = 0:0.01:2*pi;
x = h+r*cos(theta);
y = k+r*sin(theta);

plot(x,y);
axis('equal');   xlabel('x-axis');    ylabel('y-axis');
\end{minted}

\subsection{Parabola}
\begin{minted}[frame=lines,framerule=3pt, framesep=10pt, fontsize=\small, linenos]{matlab}
fprintf('A Parabola in parametric-from.\n\n');

a = input('Enter semi-major axis:');
h = input('Enter x-coordinate of center:');
k = input('Enter y-coordinate of center:');

t= -4:0.1:-4;
x = 2*a*t;
y = a*t.^2;    plot(x,y)
\end{minted}
\clearpage

\subsection{Ellipse}
\begin{minted}[frame=lines,framerule=3pt, framesep=10pt, fontsize=\small, linenos]{matlab}
fprintf('A Ellispe in parametirc-form.\n\n');

a = input('Enter the semi-major axis:');
b = input('Enter the semi-minor axis:');
h = input('Enter x-coordinate of center:');
k = input('Enter y-coordinate of center:');

theta = 0:0.01:2*pi;
x = h+a*cos(theta);
y = k+b*sin(theta);

plot(x,y);
axis('equal');
xlabel('x-axis');
ylabel('y-axis');
\end{minted}

\subsection{Hyperbola}
\begin{minted}[frame=lines,framerule=3pt, framesep=10pt, fontsize=\small, linenos]{matlab}
fprintf('A Hyperbola in parametirc-form.\n\n');

a = input('Enter the semi-major axis:');
b= input('Enter the semi-minor axis:');
h = input('Enter x-coordinate of center:');
k = input('Enter y-coordinate of center:');

theta = (-pi/3):0.1:(pi/3);
x = h+a*sec(theta);
y = k+b*tan(theta);

plot(x,y);
axis('equal');
xlabel('x-axis');
ylabel('y-axis');
\end{minted}



\section{Plotting}

\subsection{Multi-plot}
\begin{minted}[frame=lines,framerule=3pt, framesep=10pt, fontsize=\small, linenos]{matlab}
x = -2:0.1:4;
y = 3*x.^3-26*x+6;
yd = 9*x.^2-26;
ydd = 18*x;

plot(x,y,'-b');

hold on
plot(x,yd,'-.k');
plot(x,ydd,'--m');
title('Fucntion and its Derivatives');
xlabel('x-axis');
ylabel('y_axis');
legend('Function', 'First-Deri', 'Second-Deri')
hold off
\end{minted}

\vspace{5mm}

\subsection{Sub-Plot}
\begin{minted}[frame=lines,framerule=3pt, framesep=10pt, fontsize=\small, linenos]{matlab}
x = -2:0.1:4;
y = 3*x.^3-26*x+6;

yd = 9*x.^2-26;
ydd = 18*x;

subplot(1,3,1), plot(x,y), xlabel('x-axis'),
 ylabel('y-axis'), %1-row 3-column 1-positon.
subplot(1,3,2), plot(x,yd);
subplot(1,3,3), plot(x,ydd);
\end{minted}
\clearpage


\subsection{3D-Plot}
\begin{minted}[frame=lines,framerule=3pt, framesep=10pt, fontsize=\small, linenos]{matlab}
x = -2:0.01:4;
y = sin(x);
z = x.^2 + y.^2;

plot3(x,y,z);
xlabel('x');     ylabel('y');    zlabel('z');
title('Curve in 3D');
\end{minted}

\vspace{25mm}

\subsection{Surface-Plot}
\begin{minted}[frame=lines,framerule=3pt, framesep=10pt, fontsize=\small, linenos]{matlab}
x = -2:0.01:4;
y = -3:0.01:4;
[u,v] = meshgrid(x,y);

z = sin(u) + cos(v);

surf(x,y,z);

xlabel('x');     ylabel('y');    zlabel('z');
title('Surface Plotting');
shading interp;
colorbar;
\end{minted}
