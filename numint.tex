\section{Code for Trapezoidal Rule}
\begin{minted}[frame=lines,framerule=3pt, framesep=10pt, fontsize=\small, linenos]{matlab}

% Initialization
f=@(x) x*sin(x);

a=0;     b=(pi/2);
n=5;     h=(b-a)/n;

S=0.5*(f(a)+f(b));
S1=0;

% Scheme
for i=1:n-1
  xi=a+i*h;
  S1=S1+f(xi);
end

I=h*(S+S1);

fprintf('The integral is %f\n',I)

\end{minted}
\(ans=1.008265\)


\section{Code for Simpson's 1/3 rule}
\begin{minted}[frame=lines,framerule=3pt, framesep=10pt, fontsize=\small, linenos]{matlab}

%Initialization
f=@(x) x*sin(x);

a=0;     b=(pi/2);
n = 12;  h=(b-a)/n;

S=f(a)+f(b);
S1=0;
S2=0;

% Scheme
for i=1:2:n-1
  xi=a+i*h;
  S1=S1+4*f(xi);
end

for i=2:2:n-2
  xi=a+i*h;
  S2=S2+2*f(xi);
end

% Output
I=(h/3)*(S+S1+S2);
fprintf('The integral value is %f.\n',I)

\end{minted}
\(ans=0.999995\)

\clearpage
\section{Code for Simpson's 3/8 rule}
\begin{minted}[frame=lines,framerule=3pt, framesep=10pt, fontsize=\small, linenos]{matlab}

% Initialization
f=@(x) x*sin(x);
a=0;           b=pi/2;     n=12;     h=(b-a)/n;

S1=f(a)+f(b);  S2=0;     S3=0;

% Scheme
for i=1:3:n-2
  x1=a+i*h;
  x2=a+(i+1)*h;
  S2=S2+f(x1)+f(x2);
end

for i=3:3:n-3
  S3=S3+f(a+i*h);
end

% Output
I=(3*h/8)*(S1+3*S2+2*S3);
fprintf('The integral value is %f.\n',I)

\end{minted}

\(ans=0.999989\)
